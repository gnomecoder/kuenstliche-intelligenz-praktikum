\documentclass[12pt, a4paper]{article}

\usepackage{german}
\usepackage[utf8]{inputenc}
\usepackage[T1]{fontenc}
\usepackage{amsmath}
\usepackage{amssymb}
\usepackage{hyperref}
\usepackage{marvosym}

\newcommand{\changefont}[3]{
\fontfamily{#1} \fontseries{#2} \fontshape{#3} \selectfont}

\usepackage{tikz}
\usetikzlibrary{trees}

\usepackage{bigstrut}

\title{KI - Praktikum2}
\author{Gruppe A\_Blau\_WS1415\\\\
    Felix Gebauer\\
    Steffen Lang\\
    Mara Braun\\
    Christoph Hegemann\\
    Janis Saritzoglou}


\begin{document}
\maketitle
\newpage
\section*{Aufgabe 1}
Gene:
\begin{itemize}
\item A: 500
\item B: 10
\item C: 0
\item D: -200
\end{itemize}
8 Gene pro Individuum. Startpopulation 500 zufallsgenerierte
Individuen.

\subsection*{a}
Sei $\mathbb{G} = \{500,10,0,-200\}$ die Menge aller Gene, dann ist
$\mathbb{I} = \mathbb{G}^8$ die Menge aller möglichen Individuen mit 8
Genen.

\subsubsection*{Fitnessfunktion}
Sei $P \subseteq \mathbb{I}$ eine Population.\\
Dann weist die Fitnessfunktion $F$ einem jeden Individuum $i \in P$
einen Wert zu.\\
\begin{align*}
F: \mathbb{I} \longrightarrow \mathbb{Z}\\
F(i) = \sum_j i_j
\end{align*}

\subsubsection*{Abbruchbedingung}
Wir sehen für die Abbruchbedingung zwei sinnvolle Möglichkeiten:
\begin{itemize}
\item Abbruch nach Erreichen eines gegebenen Zielwertes
\item Abbruch nach fehlender Verbesserung über mehrere Generationen
\end{itemize}
In jedem Fall muss nach einer maximalen Anzahl von Evolutionsschritten
abgebrochen werden.

\subsection*{b}
Der genetische Code für ein perfektes Individuum lautet: <1,1,1,1,1,1,1,1>
Das Gen 1 mit dem Fitnesswert +500 ist für unsere Zielfunktion das beste Gen. Dementsprechend ist das Individuum, welches acht mal das Gen 1 beinhaltet das optimale Individuum.

\subsection*{c}
Wie hoch ist die Wahrscheinlichkeit, dass dieses Individuum in unserer zufällig generierten ersten Generation bereits einmal enthalten ist?\\
\begin{align*}
i_{max} &= max(F\mathbb(I)) \\\\
P(i_{max}\in Population) &= \frac{\text{Anzahl der günstigen Individuen * Populationsgröße}}{\text{Anzahl der möglichen Individuen}} \\\\
P(i_{max}\in Population) &= \frac{1*500}{4^8} = 0.7629\%
\end{align*}

\subsection*{e}
Welche Gefahr bestünde, wenn wir auf den Mutationsschritt verzichten würden?\\\\
Unsere gegebenen Möglichkeiten neue Individuen zu gewinnen, sind \textit{1-Punkt-Crossover} und \textit{Random Resetting}.\\
Beim 1 Punkt Crossover kann kein Kind Gene enthalten, die nicht an der selben Position bei einem der Elternteile vorkommen. Lässt man die Mutation (\textit{Random Resetting}) nun weg, kann eine Population ohne Individuum mit Gen $X$ an der Stelle $n$ nie ein Individuum mit Gen $X$ an der Stelle $n$ hervorbringen. Das hätte zur Folge, dass bei einer ungünstigen Startpopulation ein Erreichen des vollständigen Lösungsraums nicht möglich ist.

\subsection*{f}
Welche Änderung würden Sie vornehmen, um den oben beschriebenen Algorithmus zu optimieren?\\\
Statt bei der Reproduktion auf eine zufällige Ersetzung durch ``Steady
State Full Random'' zu setzen, ist es sinnvoller ``Full Replacement''
anzuwenden. Da die Problemstellung mit einem ``Greedy'' Ansatz gelöst
werden kann, ist ein schnelles, gieriges Vorgehen hier geeignet.

\section*{Aufgabe 2}
\subsubsection*{b)}
\textit{Dokumentieren und begründen sie ihre Designentscheidungen.}

Wir haben uns für folgende Konfiguration entschieden:

\begin{itemize}
\item Populationsgröße: \textbf{5}
\item Gene pro Individuum: \textbf{8}
Diese hohe Anzahl an Genen pro Individuum erlaubt das Lernen und den
Evolutionsgedanken zu verstärken. Viele Gene erlauben interessante Rekombinationsmöglichkeiten.
\item Anzahl unterschiedlicher Gene: \textbf{4}\\
Wir haben uns für diese geringe Anzahl an unterschiedlichen Genen
entschieden, um den Lösungsraum bewusst zu verkleinern. Und Redundanz
bei der Genauswahl zu vermeiden.
\item Selektionsalgorithmus: \textbf{Gewichtete Wahrscheinlichkeit}
\item Rekombination: \textbf{1-Punkt-Crossover}
\item Mutation: \textbf{Random-Resetting}\\
Die vorherigen 3 Punkte sind aus Aufgabe 1 übernommen, und
\textit{keine} Designentscheidung. Der Vollständigkeit halber sollen sie hier
dennoch aufgelistet werden.
\item Reproduktion: \textbf{Steady-State Strongest for Weakest}\\
Wir haben uns für diese Reproduktionsmethode entschieden, da wir bei
unserer vergleichsweise kleinen Populationsgröße, schnelle
Verbesserungen benötigen.

\end{itemize}




\end{document}

%%% Local Variables:
%%% mode: latex
%%% TeX-master: t
%%% End:
