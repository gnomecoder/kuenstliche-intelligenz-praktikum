\documentclass[12pt, a4paper]{article}

\usepackage{german}
\usepackage[utf8]{inputenc}
\usepackage[T1]{fontenc}

\usepackage{tikz}
\usetikzlibrary{trees}

\usepackage{bigstrut}

\title{KI - Praktikum1}
\author{Gruppe A\_Blau\_WS1415}


\begin{document}

%%%%% Styling für Bäume %%%%%%%%%
% Set the overall layout of the tree
\tikzstyle{level 1}=[level distance=3.5cm, sibling distance=3.5cm]
\tikzstyle{level 2}=[level distance=3.5cm, sibling distance=2cm]

% Define styles for bags and leafs
\tikzstyle{bag} = [text width=4em, text centered]
\tikzstyle{end} = [circle, minimum width=3pt,fill, inner sep=0pt]
%%%%%%%%%%%%%%%%%%%%%%%%%%%%%%%%%

\maketitle
\newpage
\section{}
Der Campus Gummersbach der FH Köln hat 1000 Studierende. Es wurde eine Umfrage unter
allen Studierenden gemacht. Die folgende Matrix gibt Aufschluss darüber, wie die
Raucher/innen auf die Studierenden verteilt sind.\\\\

\begin{tabular}[bct]{|c|c|c|c|}
\hline \bigstrut
& \textbf{B:} weiblich & \textbf{$\neg$B:} männlich &\textbf{Summe}\\
\hline \bigstrut
\textbf{A:} Raucher & 410 & 397 & 807 \\
\hline \bigstrut
\textbf{$\neg$A:} Nichtraucher & 114 & 79 & 193 \\
\hline \bigstrut
\textbf{Summe} & 524 & 476 & 1000\\ \hline
\end{tabular}
\subsection*{a)}
Berechnen Sie die relativen Häufigkeiten und tragen Sie sie in eine
neue Matrix ein.\\\\

\begin{tabular}[bct]{|c|c|c|c|}
\hline \bigstrut
& \textbf{B:} weiblich & \textbf{$\neg$B:} männlich &\textbf{Summe}\\
\hline \bigstrut
\textbf{A:} Raucher & 0,410 & 0,397 & 0,807 \\
\hline \bigstrut
\textbf{$\neg$A:} Nichtraucher & 0,114 & 0,079 & 0,193 \\
\hline \bigstrut
\textbf{Summe} & 0,524 & 0,476 & 1\\ \hline
\end{tabular}
\subsection*{b)}
Die neue Matrix enthält unsere a priori Wahrscheinlichkeiten und entspricht der Vierfeldtafel
für P(A|B), P($\neg$A|B), P(A|$\neg$B), P($\neg$A|$\neg$B) (+ Summe), leiten Sie daraus die
Pfadwahrscheinlichkeiten her und zeichnen Sie sie in Baumform. Zeichnen Sie auch die
Inversion dieses Baumes.

\begin{tikzpicture}[grow=right, sloped]
\node[bag] {Start}
    child {
        node[bag] {$P(\neg R) $}
            child {
                node[end, label=right:
                    {$P(\neg R\cap M)=0,193 \times 0,524$}] {}
                edge from parent
                node[above] {$M$}
                node[below]  {$0,476$}
            }
            child {
                node[end, label=right:
                    {$P(\neg R \cap W)=0,193 \times 0,524$}] {}
                edge from parent
                node[above] {$W$}
                node[below]  {$0,524$}
            }
            edge from parent
            node[above] {$NR$}
            node[below]  {$0,193$}
    }
    child {
        node[bag] {$P(R)$}
        child {
                node[end, label=right:
                    {$P(R\cap M)=0,807\times 0,476$}] {}
                edge from parent
                node[above] {$M$}
                node[below]  {$0,476$}
            }
            child {
                node[end, label=right:
                    {$P(R\cap W)=0,807\times0,524 $}] {}
                edge from parent
                node[above] {$W$}
                node[below]  {$0,524$}
            }
        edge from parent
            node[above] {$R$}
            node[below]  {$0,807$}
    };
\end{tikzpicture}
\end{document}
