\documentclass[12pt, a4paper]{article}

\usepackage{german}
\usepackage[utf8]{inputenc}
\usepackage[T1]{fontenc}
\usepackage{amsmath}
\usepackage{hyperref}
\usepackage{marvosym}

\newcommand{\changefont}[3]{
\fontfamily{#1} \fontseries{#2} \fontshape{#3} \selectfont}

\usepackage{tikz}
\usetikzlibrary{trees}

\usepackage{bigstrut}

\title{KI - Praktikum1}
\author{Gruppe A\_Blau\_WS1415\\\\
    Felix Gebauer\\
    Steffen Lang\\
    Mara Braun\\
    Christoph Hegemann\\
    Janis Saritzoglou}


\begin{document}

%Schriftart einstellbar
%\changefont{pbk}{m}{n}

%%%%% Styling für Bäume %%%%%%%%%
% Set the overall layout of the tree
\tikzstyle{level 1}=[level distance=3.5cm, sibling distance=3.5cm]
\tikzstyle{level 2}=[level distance=3.5cm, sibling distance=2cm]

% Define styles for bags and leafs
\tikzstyle{bag} = [text width=4em, text centered]
\tikzstyle{end} = [circle, minimum width=3pt,fill, inner sep=0pt]
%%%%%%%%%%%%%%%%%%%%%%%%%%%%%%%%%

\maketitle
\newpage
\section{}
Der Campus Gummersbach der FH Köln hat 1000 Studierende. Es wurde eine Umfrage unter
allen Studierenden gemacht. Die folgende Matrix gibt Aufschluss darüber, wie die
Raucher/innen auf die Studierenden verteilt sind.\\\\

\begin{tabular}[bct]{|c|c|c|c|}
\hline \bigstrut
& \textbf{B:} weiblich & \textbf{$\neg$B:} männlich &\textbf{Summe}\\
\hline \bigstrut
\textbf{A:} Raucher & 410 & 397 & 807 \\
\hline \bigstrut
\textbf{$\neg$A:} Nichtraucher & 114 & 79 & 193 \\
\hline \bigstrut
\textbf{Summe} & 524 & 476 & 1000\\ \hline
\end{tabular}
\subsection*{a)}
Berechnen Sie die relativen Häufigkeiten und tragen Sie sie in eine
neue Matrix ein.\\\\

\begin{tabular}[bct]{|c|c|c|c|}
\hline \bigstrut
& \textbf{B:} weiblich & \textbf{$\neg$B:} männlich &\textbf{Summe}\\
\hline \bigstrut
\textbf{A:} Raucher & 0,410 & 0,397 & 0,807 \\
\hline \bigstrut
\textbf{$\neg$A:} Nichtraucher & 0,114 & 0,079 & 0,193 \\
\hline \bigstrut
\textbf{Summe} & 0,524 & 0,476 & 1\\ \hline
\end{tabular}
\subsection*{b)}
Die neue Matrix enthält unsere a priori Wahrscheinlichkeiten und entspricht der Vierfeldtafel
für $P(A|B)$, $P(\neg A|B)$ $P(A | \neg B)$, $P(\neg A|\neg B)$ (+ Summe), leiten Sie daraus die
Pfadwahrscheinlichkeiten her und zeichnen Sie sie in Baumform. Zeichnen Sie auch die
Inversion dieses Baumes.
\vspace{0.3cm}

\begin{tikzpicture}[grow=right, sloped]
\node[bag] {Start}
    child {
        node[bag] {$P(\neg A) $}
            child {
                node[end, label=right:
                    {$P(\neg A\cap \neg B)=0,193 \times 0,476$}] {}
                edge from parent
                node[above] {$\neg B$}
                node[below]  {$0,476$}
            }
            child {
                node[end, label=right:
                    {$P(\neg A \cap B)=0,193 \times 0,524$}] {}
                edge from parent
                node[above] {$B$}
                node[below]  {$0,524$}
            }
            edge from parent
            node[above] {$\neg A$}
            node[below]  {$0,193$}
    }
    child {
        node[bag] {$P(A)$}
        child {
                node[end, label=right:
                    {$P(A\cap \neg B)=0,807\times 0,476$}] {}
                edge from parent
                node[above] {$\neg B$}
                node[below]  {$0,476$}
            }
            child {
                node[end, label=right:
                    {$P(A\cap B)=0,807\times0,524 $}] {}
                edge from parent
                node[above] {$B$}
                node[below]  {$0,524$}
            }
        edge from parent
            node[above] {$A$}
            node[below]  {$0,807$}
    };
\end{tikzpicture}
\subsection*{c)}
Berechnen Sie die fehlenden bedingten Wahrscheinlichkeiten und tragen Sie diese in die Baumdiagramme aus \textbf{b)} ein.
\begin{align}
P(A | B) = \frac{P(A \cap B)}{P(B)} = \frac{0,410}{0,524} = 0,782 \\
P(\neg A | B) = 1 - P(A | B) = 1 - 0,782 = 0,218 \\
P(A | \neg B) = \frac{P(A \cap \neg B)}{P(\neg B)} = \frac{0,397}{0,476} = 0,834 \\
P(\neg A | \neg B) = 1 - P(A | \neg B) = 1 - 0,834 = 0,166 \\\\
P(B | A) = \frac{P(B \cap A)}{P(A)} = \frac{0,410}{0,807} = 0,508 \\
P(\neg B | A) = 1 - P(B | A) = 1 - 0,508 = 0,492 \\
P(\neg B | \neg A) = \frac{P(\neg B \cap \neg A)}{P(\neg A)} = \frac{0,079}{0,193} = 0,409 \\
P(B | \neg A) = 1 - P(\neg B | \neg A) = 1 - 0,409 = 0,591
\end{align}
\vspace{0.3cm}
\subsection*{d)}
Aus der Gesamtheit aller Studierenden wird eine Person zufaellig ausgewaehlt.
\begin{enumerate}
\item Mit welcher Wahrscheinlichkeit raucht die Person nicht?\\
\begin{align}
P(\neg A) = 0,193 \, \widehat{=} \, 19,3\%
\end{align}
\item Mit welcher Wahrscheinlichkeit ist die Person weiblich?
\begin{align}
P(B) = 0,524 \, \widehat{=} \, 52,4\%
\end{align}
\item Falls die Person nicht raucht, mit welcher Wahrscheinlichkeit ist sie maennlich?
\begin{align}
P(\neg B | \neg A) = \frac{P(\neg B \cap \neg A)}{P(\neg A)} = \frac{0,079}{0,193} = 0,409
\end{align}
\item Falls die Person weiblich ist, mit welcher Wahrscheinlichkeit raucht sie?
\begin{align}
P(A | B) = \frac{P(A \cap B)}{P(B)} = \frac{0,410}{0,524} = 0,782
\end{align}
\item Besteht ein Zusammenhang zwischen Geschlecht und Rauchen?
\begin{align}
\frac{P(A | B)}{P(\neg A | B)} = \frac{P(A | \neg B)}{P(\neg A | \neg B)} \\
\frac{0,782}{P(\neg A | B)} = \frac{P(A | \neg B)}{0,166}
\\
P(A | B) = 0,782 \Rightarrow P(\neg A | B) = 1 - 0,782 = 0,218
\\
P(\neg A | \neg B) = 0,166 \Rightarrow P(\neg A | B) = 1 - 0,166 = 0,834
\\
\frac{0,782}{0,218} = \frac{0,834}{0,166}
\text{\Huge{\Lightning}}
\end{align}
\\
$\Rightarrow$ Verhältnismäßig existieren mehr männliche Raucher als weibliche. (Linke Seite der Gleichung beschreibt Weibliche Stundenten, die Rechte Seite die männlichen Studenten.)
\end{enumerate}
\subsection*{e)}
Erstellen Sie das Kausale Netz in SamIam \url{http://reasoning.cs.ucla.edu/samiam/} und tragen Sie die entsprechenden Wahrscheinlichkeiten ein.
\vspace{0.3cm}


\maketitle
\newpage
\section{}
Betrachten Sie eine Welt, die durch die Zufallsvariablen \textbf{(A)}Wetter, \textbf{(B)}Zahnloch und \textbf{(C)}Zahnschmerzen vollständig beschrieben wird. Unsere hypothetische Umfragepopulation besteht aus 90-jährigen Kriegsveteranen mit grässlichen Zähnen, einige von ihnen besitzen außerdem die übernatürliche Fähigkeit, Tiefdruckgebiete durch plötzlich eintretende Zahnschmerzen zu erkennen (der Rest benutzt für diese Zwecke eine alte Knieverletzung). Der Sinn unseres Netzes ist es, das Wetter zu bestimmen ohne aus dem Fenster schauen zu müssen – Stattdessen werden unsere Veteranen telefonisch befragt, sagen wir als ABM-Maßnahme für Umfrageinstitute.
Die Variablen haben folgende mögliche Zustände:
\begin{itemize}
\item P(A): sonnig, wolkig, regnerisch
\item P(B): ja / nein
\item P(C): ja / nein
\end{itemize}


\subsection*{a)}
Entscheiden Sie über die Abhängigkeiten dieser Variablen untereinander, erstellen Sie das entsprechende Kausale Netz in SamIam und begründen Sie Ihre Entscheidung.
\\\\
Die Zufallsvariablen \textbf{Wetter} und \textbf{Zahnloch} sind unabhängige Variablen, die den \textbf{Zahnschmerz} der Kriegsveteranen beeinflussen.



\subsection*{b)}
Zeichnen Sie die vollständige gemeinsame Verteilung in Form einer 3-dimensionalen Matrix, füllen Sie die Matrix mit Wahrscheinlichkeitswerten (selbst gewählt).
\\
\hspace{-0.5cm}
\begin{flushleft}
\begin{tabular}[bct]{|c|c|c|c|c|c|}
\hline \bigstrut
& \multicolumn{2}{|c|}{\textbf{C:} Zahnschmerz } & \multicolumn{2}{|c|}{\textbf{$\neg$C}: Zahnschmerz } & \textbf{Summe} \\
\hline \bigstrut
& \textbf{B:} Zahnloch & \textbf{$\neg$B:} Zahnloch & \textbf{B:} Zahnloch & \textbf{$\neg$B:} Zahnloch & \\
\hline \bigstrut
\textbf{A:} Sonnig & 0,1 & 0,01 & 0,1 & 0,19 & 0,4 \\ 
\hline \bigstrut
\textbf{A:} Wolkig & 0,12 & 0,08 & 0,08 & 0,02 & 0,3 \\ 
\hline \bigstrut
\textbf{A:} Regnerisch & 0,25 & 0,022 & 0,006 & 0,022 & 0,3 \\ 
\hline \bigstrut
\textbf{Summe} & 0,47 & 0,112 & 0,186 & 0,232 & 1.0 \\ 
\hline
\end{tabular}
\end{flushleft}


\subsection*{c)}
Erläutern Sie anhand Ihres Netzes welche Variablen unter welchen Bedingungen voneinander abhängig sind, demonstrieren Sie geschlossene und offene Knoten in ihrem Netz.
\\\\
Die Zufallsvariable \textbf{Zahnschmerz} ist nur von \textbf{Zahnloch} und \textbf{Wetter} abhängig, wenn die Kriegsveteranen ein Zahnloch besitzen bzw. sich in einem Tiefdruckgebiet befinden. Wenn also die Sonne scheint und kein Zahnloch vorhanden ist, ist die Variable \textbf{Zahnschmerz} unabhängig.\\
Der Knoten \textbf{Zahnschmerz} ist geschlossen, wenn keine Evidenzen vorhanden sind. Der Knoten ist offen, sobald Evidenzen vorliegen.
\end{document}